\documentclass[12pt]{article}
\usepackage[utf8]{inputenc}
\usepackage{latexsym,amsfonts,amssymb,amsthm,amsmath}
\usepackage{float}
\usepackage{caption}
\usepackage{marginnote}
\usepackage{tikz}
\usepackage{hyperref}

\setlength{\parindent}{0in}
\setlength{\oddsidemargin}{0in}
\setlength{\textwidth}{6.5in}
\setlength{\textheight}{8.8in}
\setlength{\topmargin}{0in}
\setlength{\headheight}{18pt}

\newtheorem*{answer*}{Answer}
\newtheorem*{solution*}{Solution}
\newtheorem{remark}{Remark}

\title{Weekly Homework 21}
\author{Math Gecs}
\date{June 23, 2024}

\begin{document}
\maketitle

\subsection*{Exercise 1}
What is the maximum value of $n$ for which there is a set of distinct positive integers $k_1, k_2, ... k_n$ for which

\[k^2_1 + k^2_2 + ... + k^2_n = 2002?\]

$\text{(A) }14 \qquad \text{(B) }15 \qquad \text{(C) }16 \qquad \text{(D) }17 \qquad \text{(E) }18$ \\

Source: 2002 AMC 12P Problem 13\\

\begin{answer*}
$\boxed {\textbf{(D) }17}$
\end{answer*}

\begin{solution*}
Note that $k^2_1 + k^2_2 + ... + k^2_n = 2002 \geq \frac{n(n+1)(2n+1)}{6}$
\\ \\
When $n = 17$, $\frac{n(n+1)(2n+1)}{6} = \frac{(17)(18)(35)}{6} = 1785 < 2002$.
\\ \\
When $n = 18$, $\frac{n(n+1)(2n+1)}{6} = 1785 + 18^2 = 2109 > 2002$.
\\ \\
Therefore, we know $n \leq 17$.
\\ \\
Now we must show that $n = 17$ works. We replace some integer $b$ within the set $\{1, 2, ... 17\}$ with an integer $a > 17$ to account for the amount under $2002$, which is $2002-1785 = 217$.
\\ \\
Essentially, this boils down to writing $217$ as a difference of squares. Assume there exist positive integers $a$ and $b$ where $a > 17$ and $b \leq 17$ such that $a^2 - b^2 = 217$.
\\ \\
We can rewrite this as $(a+b)(a-b) = 217$. Since $217 = 7 \cdot 31$, either $a+b = 217$ and $a-b = 1$ or $a+b = 31$ and $a-b = 7$. We analyze each case separately.
\\ \\
Case 1: $a+b = 217$ and $a-b = 1$
\\ \\
Solving this system of equations gives $a = 109$ and $b = 108$. However, $108 > 17$, so this case does not yield a solution.
\\ \\
Case 2: $a+b = 31$ and $a-b = 7$
\\ \\
Solving this system of equations gives $a = 19$ and $b = 12$. This satisfies all the requirements of the problem.
\\ \\
The list $1, 2 ... 11, 13, 14 ... 17, 19$ has $17$ terms whose sum of squares equals $2002$. Since $n \geq 18$ is impossible, the answer is $\boxed {\textbf{(D) }17}$.
\end{solution*}



\end{document}