\documentclass[12pt]{article}
\usepackage[utf8]{inputenc}
\usepackage{latexsym,amsfonts,amssymb,amsthm,amsmath}
\usepackage{float}
\usepackage{caption}
\usepackage{marginnote}
\usepackage{tikz}
\usepackage{hyperref}

\setlength{\parindent}{0in}
\setlength{\oddsidemargin}{0in}
\setlength{\textwidth}{6.5in}
\setlength{\textheight}{8.8in}
\setlength{\topmargin}{0in}
\setlength{\headheight}{18pt}

\newtheorem*{answer*}{Answer}
\newtheorem*{solution*}{Solution}
\newtheorem{remark}{Remark}

\title{Weekly Homework 28}
\author{Math Gecs}
\date{September 15, 2024}

\begin{document}
\maketitle

\subsection*{Exercise 1}
$\alpha$, $\beta$, and $\gamma$ are the roots of $x(x-200)(4x+1) = 1$. Let\[\omega = \tan^{-1}(\alpha) + \tan^{-1}(\beta) + \tan^{-1} (\gamma).\]The value of $\tan(\omega)$ can be written as $\tfrac{m}{n}$ where $m$ and $n$ are relatively prime positive integers. Determine the value of $m+n$.\\

Source: Mock AIME 2 Pre 2005 Problem 11\\

\begin{answer*}
$\boxed{167}$
\end{answer*}

\begin{solution*}
We know that $\alpha, \beta, \gamma$ are the roots of $x(x-200)(x+1/4)-1/4 = x^3 - \frac{799}{4}x^2 - 50x - \frac{1}{4}$. By Vieta's formulas, we have:

$\alpha + \beta + \gamma = \frac{799}{4}$

$\alpha\beta + \beta\gamma + \gamma\alpha = -50$

$\alpha\beta\gamma = \frac{1}{4}$

Now, by tangent addition formulas, we have $\tan(\omega) = \frac{\alpha + \beta + \gamma - \alpha\beta\gamma}{1 - \alpha\beta - \beta\gamma - \gamma\alpha}$. Substituting Vieta's formulas, we obtain $\tan(\omega) = \frac{\frac{799}{4} - \frac{1}{4}}{1 - (-50)} = \frac{\frac{798}{4}}{51} = \frac{133}{34}$. Therefore, our answer is $133 + 34 = \boxed{167}$ and we are done.
\end{solution*}


\end{document}