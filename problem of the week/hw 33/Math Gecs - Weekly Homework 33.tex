\documentclass[12pt]{article}
\usepackage[utf8]{inputenc}
\usepackage{latexsym,amsfonts,amssymb,amsthm,amsmath}
\usepackage{float}
\usepackage{caption}
\usepackage{marginnote}
\usepackage{tikz}
\usepackage{hyperref}

\setlength{\parindent}{0in}
\setlength{\oddsidemargin}{0in}
\setlength{\textwidth}{6.5in}
\setlength{\textheight}{8.8in}
\setlength{\topmargin}{0in}
\setlength{\headheight}{18pt}

\newtheorem*{answer*}{Answer}
\newtheorem*{solution*}{Solution}
\newtheorem{remark}{Remark}

\title{Weekly Homework 33}
\author{Math Gecs}
\date{October 20, 2024}

\begin{document}
\maketitle

\subsection*{Exercise 1}
Let $\gamma$ be circle and let $P$ be a point outside $\gamma$. Let $PA$ and $PB$ be the tangents from $P$ to $\gamma$ (where $A, B \in \gamma$). A line passing through $P$ intersects $\gamma$ at points $Q$ and $R$. Let $S$ be a point on $\gamma$ such that $BS \parallel QR$. Prove that $SA$ bisects $QR$.\\

Source: 2000 Pan African MO Problem 5\\

\begin{solution*}
There is a projective transformation which maps $\gamma$ to a circle and that maps the midpoint of $QR$ to its center (EXPAND); therefore, we may assume without loss of generality that the midpoint of $QR$ is the center of $\gamma$. But then $B$ is the reflection of $A$ across $QR$, so that $S$ is the antipode of $A$ on $\gamma$, and we are done.
\end{solution*}


\end{document}