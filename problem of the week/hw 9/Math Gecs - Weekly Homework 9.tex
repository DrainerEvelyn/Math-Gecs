\documentclass[12pt]{article}
\usepackage[utf8]{inputenc}
\usepackage{latexsym,amsfonts,amssymb,amsthm,amsmath}
\usepackage{float}
\usepackage{caption}
\usepackage{marginnote}
\usepackage{tikz}
\usepackage{hyperref}

\setlength{\parindent}{0in}
\setlength{\oddsidemargin}{0in}
\setlength{\textwidth}{6.5in}
\setlength{\textheight}{8.8in}
\setlength{\topmargin}{0in}
\setlength{\headheight}{18pt}

\newtheorem*{answer*}{Answer}
\newtheorem*{solution*}{Solution}
\newtheorem{remark}{Remark}

\title{Weekly Homework 9}
\author{Math Gecs}
\date{March 23, 2024}

\begin{document}
\maketitle

\subsection*{Exercise 1}
For what real values of $x$ is


$$ \sqrt{x+\sqrt{2x-1}} + \sqrt{x-\sqrt{2x-1}} = A,$$

given (a) $A=\sqrt{2}$, (b) $A=1$, (c) $A=2$, where only non-negative real numbers are admitted for square roots?
\\ \\
Source: 1959 IMO Problem 2\\

\begin{answer*}
$x= \frac{3}{2}$
\end{answer*}

\begin{solution*}
The square roots imply that $x\ge \frac{1}{2}$.

Square both sides of the given equation: $$A^2 = \Big( x + \sqrt{2x - 1}\Big) + 2 \sqrt{x + \sqrt{2x - 1}}  \sqrt{x - \sqrt{2x - 1}} +  \Big( x - \sqrt{2x - 1}\Big) $$

Add the first and the last terms to get:
$$A^2 = 2x + 2 \sqrt{x + \sqrt{2x - 1}}  \sqrt{x - \sqrt{2x - 1}}$$

Multiply the middle terms, and use $(a + b)(a - b) = a^2 - b^2$ to get:
$$A^2 = 2x + 2 \sqrt{x^2 - 2x + 1}$$

Since the term inside the square root is a perfect square, and by factoring 2 out, we get
$$A^2 = 2(x + \sqrt{(x-1)^2})$$
Use the property that $\sqrt{x^2}=|x|$ to get
$$A^2 = 2(x+|x-1|)$$

\textbf{Case I:} If $x \le 1$, then $|x-1| = 1 - x$, and the equation reduces to $A^2 = 2$. This is precisely part (a) of the question, for which the valid interval is now $x \in \left[ \frac{1}{2}, 1 \right]$
\\ \\
\textbf{Case II:} If $x > 1$, then $|x-1| = x - 1$ and we have
$$x = \frac{A^2 + 2}{4} > 1$$
which simplifies to 
$$A^2 > 2 $$

This tells there that there is no solution for (b), since we must have $A^2 \ge 2$

For (c), we have $A = 2$, which means that $A^2 = 4$, so the only solution is $ x=\frac{3}{2}$.
\end{solution*}
$$$$
\begin{solution*}
Note that the equation can be rewritten to 
$$\sqrt{(\sqrt{2x-1}+1)^2} + \sqrt{(\sqrt{2x-1}-1)^2}=A\sqrt{2}$$
i.e., $\sqrt{2x-1}+1 + |\sqrt{2x-1}-1|=A\sqrt{2}$. 
\\ \\
\textbf{Case I:} when $2x-1\ge 1$ (i.e., $x\ge 1$), the equation becomes $2\sqrt{2x-1}=\sqrt{2}A$. For (a), we have $x=1$; for (b) we have $x=\frac{3}{4}$; for (c) we have $x=\frac{3}{2}$. Since $x\ge 1$, (b) $x=\frac{3}{4}$ is not what we want.
\\ \\
\textbf{Case II:} when $0\le 2x-1 <1$ (i.e., $1/2\le x <1$), the equation becomes $2=\sqrt{2}A$, which only works for (a) $A=\sqrt{2}$. 
\\ \\
In summary, any $x \in \left[\frac{1}{2}, 1\right]$ is a solution for (a); there is no solution for (b); there is one solution for (c), which is $x=\frac{3}{2}$. 
\end{solution*}

\vspace{3.1in}






\subsection*{Exercise 2}
Let $\{a_n \}_{n\ge 0}$ be a non-decreasing, unbounded sequence of non-negative integers with $a_0=0$.  Let the number of members of the sequence not exceeding $n$ be $b_n$.  Prove that for all positive integers $m$ and $n$, we have
$$ a_0 + a_1 + \dotsb + a_m + b_0 + b_1 + \dotsb + b_n \ge (m+1)(n+1) . $$\\

Source: 1999 BMO Problem 4\\

$$$$


\begin{proof}
Note that for arbitrary nonnegative integers $i,j$, the relation $j \le a_i$ is equivalent to the relation $i \ge b_{j-1}$.  It then follows that
$$ \sum_{i=0}^m a_i = \sum_{i=0}^m \sum_{j=1}^{a_i} 1 = \sum_{j=1}^{a_m} \sum_{i=b_{j-1}}^m 1 = \sum_{j=1}^{a_m} ( m+1 - b_{j-1} ) = \sum_{j=0}^{a_m-1} (m+1 - b_j ) . $$
Note that if $j \le a_m-1$, then there are at most $m$ terms of $\{ a_k\}_{k\ge 0}$ which do not exceed $j$, i.e., $b_j \le m$; it follows that every term of the last summation is positive.

Now, if $a_m \ge n+1$, then we have
$$ \begin{align*}
\sum_{i=0}^m a_i + \sum_{j=0}^n b_j &= \sum_{j=n+1}^{a_m-1}(m+1 - b_j) + \sum_{j=0}^n (m+1 - b_j + b_j) \\
&= \sum_{j=n+1}^{a_m-1}(m+1-b_j) + (n+1)(m+1) \ge (n+1)(m+1),
\end{align*} $$
as desired.  On the other hand, if $a_m < n+1$, then for all $j\ge a_m$, $b_j \ge m+1$.  It then follows that
$$ \begin{align*}
 \sum_{i=0}^m a_j + \sum_{j=0}^n b_j &= \sum_{j=0}^{a_m-1}(m+1 - b_j + b_j) + \sum_{j=a_m}^n b_j \\
&= (a_m)(m+1) + \sum_{j=a_m}^n b_j \\
&\ge (a_m)(m+1) + (n+1-a_m)(m+1) = (n+1)(m+1),
\end{align*} $$
as desired.  Therefore the problem statement is true in all cases.

\end{proof}


\end{document}