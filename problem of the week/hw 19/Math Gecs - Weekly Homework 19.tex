\documentclass[12pt]{article}
\usepackage[utf8]{inputenc}
\usepackage{latexsym,amsfonts,amssymb,amsthm,amsmath}
\usepackage{float}
\usepackage{caption}
\usepackage{marginnote}
\usepackage{tikz}
\usepackage{hyperref}

\setlength{\parindent}{0in}
\setlength{\oddsidemargin}{0in}
\setlength{\textwidth}{6.5in}
\setlength{\textheight}{8.8in}
\setlength{\topmargin}{0in}
\setlength{\headheight}{18pt}

\newtheorem*{answer*}{Answer}
\newtheorem*{solution*}{Solution}
\newtheorem{remark}{Remark}

\title{Weekly Homework 19}
\author{Math Gecs}
\date{June 08, 2024}

\begin{document}
\maketitle

\subsection*{Exercise 1}
The nine horizontal and nine vertical lines on an $8\times8$ checkerboard form $r$ rectangles, of which $s$ are squares. The number $s/r$ can be written in the form $m/n,$ where $m$ and $n$ are relatively prime positive integers. Find $m + n.$\\

Source: 1997 AIME Problem 2\\

\begin{answer*}
$\boxed{125}$
\end{answer*}

\begin{solution*}
To determine the two horizontal sides of a rectangle, we have to pick two of the horizontal lines of the checkerboard, or ${9\choose 2} = 36$. Similarly, there are ${9\choose 2}$ ways to pick the vertical sides, giving us $r = 1296$ rectangles.
\\ \\
For $s$, there are $8^2$ unit squares, $7^2$ of the $2\times2$ squares, and so on until $1^2$ of the $8\times 8$ squares. Using the sum of squares formula, that gives us $s=1^2+2^2+\cdots+8^2=\dfrac{(8)(8+1)(2\cdot8+1)}{6}=12*17=204$.
\\ \\
Thus $\frac sr = \dfrac{204}{1296}=\dfrac{17}{108}$, and $m+n=\boxed{125}$.
\end{solution*}

\end{document}