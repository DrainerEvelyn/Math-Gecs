\documentclass[12pt]{article}
\usepackage[utf8]{inputenc}
\usepackage{latexsym,amsfonts,amssymb,amsthm,amsmath}
\usepackage{float}
\usepackage{caption}
\usepackage{marginnote}
\usepackage{tikz}
\usepackage{hyperref}

\setlength{\parindent}{0in}
\setlength{\oddsidemargin}{0in}
\setlength{\textwidth}{6.5in}
\setlength{\textheight}{8.8in}
\setlength{\topmargin}{0in}
\setlength{\headheight}{18pt}

\newtheorem*{answer*}{Answer}
\newtheorem*{solution*}{Solution}
\newtheorem{remark}{Remark}

\title{Weekly Homework 42}
\author{Math Gecs}
\date{December 24, 2024}

\begin{document}
\maketitle

\subsection*{Exercise 1}
Let b be a real number randomly selected from the interval $[-17,17]$. Then, m and n are two relatively prime positive integers such that m/n is the probability that the equation $x^4+25b^2=(4b^2-10b)x^2$ has $\textit{at least}$ two distinct real solutions. Find the value of $m+n$.\\

Source: 2007 iTest Problem 36\\

\begin{solution*}
The equation has quadratic form, so complete the square to solve for x.

\[x^4 - (4b^2 - 10b)x^2 + 25b^2 = 0\]\[x^4 - (4b^2 - 10b)x^2 + (2b^2 - 5b)^2 - 4b^4 + 20b^3 = 0\]\[(x^2 - (2b^2 - 5b))^2 = 4b^4 - 20b^3\]
In order for the equation to have real solutions,

\[16b^4 - 80b^3 \ge 0\]\[b^3(b - 5) \ge 0\]\[b \le 0 \text{ or } b \ge 5\]
Note that $2b^2 - 5b = b(2b-5)$ is greater than or equal to $0$ when $b \le 0$ or $b \ge 5$. Also, if $b = 0$, then expression leads to $x^4 = 0$ and only has one unique solution, so discard $b = 0$ as a solution. The rest of the values leads to $b^2$ equalling some positive value, so these values will lead to two distinct real solutions.

Therefore, in interval notation, $b \in [-17,0) \cup [5,17]$, so the probability that the equation has at least two distinct real solutions when $b$ is randomly picked from interval $[-17,17]$ is $\frac{29}{34}$. This means that $m+n = \boxed{63}$.
\end{solution*}


\end{document}