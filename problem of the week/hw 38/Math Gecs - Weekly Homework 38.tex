\documentclass[12pt]{article}
\usepackage[utf8]{inputenc}
\usepackage{latexsym,amsfonts,amssymb,amsthm,amsmath}
\usepackage{float}
\usepackage{caption}
\usepackage{marginnote}
\usepackage{tikz}
\usepackage{hyperref}

\setlength{\parindent}{0in}
\setlength{\oddsidemargin}{0in}
\setlength{\textwidth}{6.5in}
\setlength{\textheight}{8.8in}
\setlength{\topmargin}{0in}
\setlength{\headheight}{18pt}

\newtheorem*{answer*}{Answer}
\newtheorem*{solution*}{Solution}
\newtheorem{remark}{Remark}

\title{Weekly Homework 38}
\author{Math Gecs}
\date{November 26, 2024}

\begin{document}
\maketitle

\subsection*{Exercise 1}
An $m\times n\times p$ rectangular box has half the volume of an $(m + 2)\times(n + 2)\times(p + 2)$ rectangular box, where $m, n,$ and $p$ are integers, and $m\le n\le p.$ What is the largest possible value of $p$?\\

Source: 1998 AIME Problem 14\\


\begin{solution*}
\[2mnp = (m+2)(n+2)(p+2)\]
Let’s solve for $p$:

\[(2mn)p = p(m+2)(n+2) + 2(m+2)(n+2)\]\[[2mn - (m+2)(n+2)]p = 2(m+2)(n+2)\]\[p = \frac{2(m+2)(n+2)}{mn - 2n - 2m - 4} = \frac{2(m+2)(n+2)}{(m-2)(n-2) - 8}\]
Clearly, we want to minimize the denominator, so we test $(m-2)(n-2) - 8 = 1 \Longrightarrow (m-2)(n-2) = 9$. The possible pairs of factors of $9$ are $(1,9)(3,3)$. These give $m = 3, n = 11$ and $m = 5, n = 5$ respectively. Substituting into the numerator, we see that the first pair gives $130$, while the second pair gives $98$. We now check that $130$ is optimal, setting $a=m-2$, $b=n-2$ in order to simplify calculations. Since\[0 \le (a-1)(b-1) \implies a+b \le ab+1\]We have\[p = \frac{2(a+4)(b+4)}{ab-8} = \frac{2ab+8(a+b)+32}{ab-8} \le \frac{2ab+8(ab+1)+32}{ab-8} = 10 + \frac{120}{ab-8} \le 130\]Where we see $(m,n)=(3,11)$ gives us our maximum value of $\boxed{130}$.

Note that $0 \le (a-1)(b-1)$ assumes $m,n \ge 3$, but this is clear as $\frac{2m}{m+2} = \frac{(n+2)(p+2)}{np} > 1$ and similarly for $n$.
\end{solution*}


\end{document}