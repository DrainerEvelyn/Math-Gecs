\documentclass[12pt]{article}
\usepackage[utf8]{inputenc}
\usepackage{latexsym,amsfonts,amssymb,amsthm,amsmath}
\usepackage{float}
\usepackage{caption}
\usepackage{marginnote}
\usepackage{tikz}
\usepackage{hyperref}

\setlength{\parindent}{0in}
\setlength{\oddsidemargin}{0in}
\setlength{\textwidth}{6.5in}
\setlength{\textheight}{8.8in}
\setlength{\topmargin}{0in}
\setlength{\headheight}{18pt}

\newtheorem*{answer*}{Answer}
\newtheorem*{solution*}{Solution}
\newtheorem{remark}{Remark}

\title{Weekly Homework 11}
\author{Math Gecs}
\date{April 7, 2024}

\begin{document}
\maketitle

\subsection*{Exercise 1}
Let $p(x) = a_nx^n + a_{n-1}x^{n-1} + \cdots + a_1x+a_0$, where the coefficients $ a_i$ are integers. If $p(0)$ and $p(1)$ are both odd, show that $p(x)$ has no integral roots.\\

1971 Canadian MO Problem 5


\begin{solution*}
Inputting $0$ and $1$ into $p(x)$, we obtain

$$p(0)=a_0$$

and

$$p(1)=a_0+a_1+a_2+\cdots+a_n$$

The problem statement tells us that these are both odd. We will keep this in mind as we begin our proof by contradiction.

Suppose for the sake of contradiction that there exist integer $m$ such that

$$p(m)=0$$

Substitution gives

$$a_nm^n + a_{n-1}m^{n-1} + \cdots + a_1m+a_0=0$$

By the Integer Root Theorem, $m$ must divide $a_0$. Since $a_0$ is odd, as shown above, $m$ must be odd. We also know that $p(m)$ must be even since it is equal to $0$. From above, we have that $a_0+a_1+a_2+\cdots+a_n$ must be odd. Since we also have that $a_0$ is odd, $a_1+a_2+a_3+\cdots+a_n$ must be even. Thus, there must be an even number of odd $a_i$ for integer $0<i<n+1$. Thus, the sum of all $a_im^i$ must be even. Then for all $a_k$ that are even for integer $0<k<n+1$ we must have the sum of all $a_km^k$ even since every $a_km^k$ is even. In conclusion, we have

$$a_nm^n + a_{n-1}m^{n-1} + \cdots + a_1m$$

even. But since $a_0$ is odd, $p(m)$ must be odd. Thus, it cannot equal $0$ and we have arrived at a contradiction. $Q.E.D.$
\end{solution*}





\subsection*{Exercise 2}
Let $a,b,c \in \left[ \frac 12, 1 \right]$. Prove that 
$$2 \leq \frac{ a+b}{1+c} + \frac{ b+c}{1+a} + \frac{ c+a}{1+b} \leq 3$$.\\

Source: 2006 Romanian NMO Problems/Grade 8/Problem 4\\

\begin{answer*}
$5\le 9\left(1-\frac{2}{a+b+c+3}\right)$
\end{answer*}

\begin{solution*}
It is easy to see that the function $f(a,b,c)=\frac{a+b}{c+1}+\frac{b+c}{a+1}+\frac{c+a}{b+1}$ is convex in each of the three variables (since each term is linear or of the form $\frac{p}{x+q}$ for each variable $x$).  Thus, its value is maximized at the endpoints.  Checking the values of $f$ for all possible values of $a,b,c$ such that $a,b,c\in \{\frac{1}{2},1\}$ yields a maximum of $3$ as desired.  

As for the minimum, we have

$$2\le \frac{a+b}{c+1}+\frac{b+c}{a+1}+\frac{c+a}{b+1}$$

$$\Leftrightarrow 5\le \frac{a+b+c+1}{c+1}+\frac{a+b+c+1}{a+1}+\frac{a+b+c+1}{b+1}$$

Applying AM-HM to the right hand side yields

$$9\left(\frac{a+b+c+3}{a+b+c+1}\right)^{-1}\le\frac{a+b+c+1}{c+1}+\frac{a+b+c+1}{a+1}+\frac{a+b+c+1}{b+1}$$

$$\Rightarrow 9\left(1-\frac{2}{a+b+c+3}\right)\le\frac{a+b+c+1}{c+1}+\frac{a+b+c+1}{a+1}+\frac{a+b+c+1}{b+1}$$

Obviously, $\frac{2}{a+b+c+3}$ is maximized when $a,b,c$ are minimized.  That is, when $a=b=c=\frac{1}{2}$.  Thus, we have that 

$$5\le 9\left(1-\frac{2}{a+b+c+3}\right)$$
\end{solution*}


\end{document}