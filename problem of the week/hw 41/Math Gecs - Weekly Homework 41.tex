\documentclass[12pt]{article}
\usepackage[utf8]{inputenc}
\usepackage{latexsym,amsfonts,amssymb,amsthm,amsmath}
\usepackage{float}
\usepackage{caption}
\usepackage{marginnote}
\usepackage{tikz}
\usepackage{hyperref}

\setlength{\parindent}{0in}
\setlength{\oddsidemargin}{0in}
\setlength{\textwidth}{6.5in}
\setlength{\textheight}{8.8in}
\setlength{\topmargin}{0in}
\setlength{\headheight}{18pt}

\newtheorem*{answer*}{Answer}
\newtheorem*{solution*}{Solution}
\newtheorem{remark}{Remark}

\title{Weekly Homework 41}
\author{Math Gecs}
\date{December 17, 2024}

\begin{document}
\maketitle

\subsection*{Exercise 1}
Real numbers $x$ and $y$ are chosen independently and uniformly at random from the interval $(0,1)$. What is the probability that $\lfloor\log_2x\rfloor=\lfloor\log_2y\rfloor$?
\\
$\textbf{(A)}\ \frac{1}{8}\qquad\textbf{(B)}\ \frac{1}{6}\qquad\textbf{(C)}\ \frac{1}{4}\qquad\textbf{(D)}\ \frac{1}{3}\qquad\textbf{(E)}\ \frac{1}{2}$\\ \\
Source: 2017 AMC 12B Problem 20\\

\begin{answer*}
$\boxed{\textbf{(D)}\frac{1}{3}}$
\end{answer*}

\begin{solution*}
First let us take the case that $\lfloor \log_2{x} \rfloor = \lfloor \log_2{y} \rfloor = -1$. In this case, both $x$ and $y$ lie in the interval $[{1\over2}, 1)$. The probability of this is $\frac{1}{2} \cdot \frac{1}{2} = \frac{1}{4}$. Similarly, in the case that $\lfloor \log_2{x} \rfloor = \lfloor \log_2{y} \rfloor = -2$, $x$ and $y$ lie in the interval $[{1\over4}, {1\over2})$, and the probability is $\frac{1}{4} \cdot \frac{1}{4} = \frac{1}{16}$. Recall that the probability that $A$ or $B$ is the case, where case $A$ and case $B$ are mutually exclusive, is the sum of each individual probability. Symbolically that's $P(A \text{ or } B \text{ or } C...) = P(A) + P(B) + P(C)...$. Thus, the probability we are looking for is the sum of the probability for each of the cases $\lfloor \log_2{x} \rfloor = \lfloor \log_2{y} \rfloor = -1, -2, -3...$. It is easy to see that the probabilities for $\lfloor \log_2{x} \rfloor = \lfloor \log_2{y} \rfloor = n$ for $-\infty < n < 0$ are the infinite geometric series that starts at $\frac{1}{4}$ and with common ratio $\frac{1}{4}$. Using the formula for the sum of an infinite geometric series, we get that the probability is $\frac{\frac{1}{4}}{1 - \frac{1}{4}} = \boxed{\textbf{(D)}\frac{1}{3}}$.
\end{solution*}



\end{document}