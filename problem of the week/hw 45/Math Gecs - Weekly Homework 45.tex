\documentclass[12pt]{article}
\usepackage[utf8]{inputenc}
\usepackage{latexsym,amsfonts,amssymb,amsthm,amsmath}
\usepackage{float}
\usepackage{caption}
\usepackage{marginnote}
\usepackage{tikz}
\usepackage{hyperref}

\setlength{\parindent}{0in}
\setlength{\oddsidemargin}{0in}
\setlength{\textwidth}{6.5in}
\setlength{\textheight}{8.8in}
\setlength{\topmargin}{0in}
\setlength{\headheight}{18pt}

\newtheorem*{answer*}{Answer}
\newtheorem*{solution*}{Solution}
\newtheorem{remark}{Remark}

\title{Weekly Homework 45}
\author{Math Gecs}
\date{January 8, 2025}

\begin{document}
\maketitle

\subsection*{Exercise 1}
Determine (with proof) whether there is a subset $X$ of the integers with the following property: for any integer $n$ there is exactly one solution of $a + 2b = n$ with $a,b \in X$\\

Source: 1996 USAMO Problem 6\\


\begin{solution*}
Start with an incomplete subset $S = (S_1, S_2, S_3, ... S_m)$, such that for any integer n, there is exactly zero or one solutions to $a + 2b = n$ with $a,b \in S$. Let $N$ be the smallest integer such that for any $S_i$, $|S_i| < N$. Note that $|S_i+2S_j| < 3N$ for any $S_i$ and $S_j$
\\ \\
Suppose $M$ is the smallest non-negative integer without a solution in $S$ yet. Clearly, $0 \le M \le 3N$. Generate $S_{m+1}$ and $S_{m+2}$ such that $S_{m+1} = -10N - M$, and $S_{m+2} = 5N + M$. Thus, we now have the solution $S_{m+1}+2S_{m+2} = M$.
\\ \\
Note: The values 10 and 5 can be replaced by any sufficiently large values such that the first is twice the second.
\\ \\
Now, we must prove that the addition of these two terms to $S$ does not result in an integer n that has two solutions. Of course, $S_{m+1} + 2S_{m+2} = M$ which previously had no solutions. Furthermore, $S_{m+1} + 2S_{m+2} = -15N - M$.
\\ \\
For any $S_i$, $-12N - M < S_{m+1}+2S_i < -8N-M$, and $-21N - 2M < S_i+2S_{m+1} < -19N - 2M$.
Since $0 \le M \le 3N$, we get that $-18N < S_{m+1}+2S_i < -8N$ and $-27N < S_i+2S_{m+1} < -19N$
Similarly, $3N + M < S_{m+2}+2S_i < 5N + M$, and $9N+2M < S_i+2S_{m+2} < 11N+2M$.
Since $0 \le M \le 3N$, we get that $3N < S_{m+2}+2S_i < 8N$ and $9N < S_i+2S_{m+2} < 17N$.
Since all of these sums (other than $M$) are either greater than $3N$ or less than $-3N$, they are all sums that previously had no solutions. Furthermore, none of these sums are duplicated, as sums of different forms are contained in disjoint ranges of integers.
\\ \\
Thus, we have proved that we can generate a subset $S$ such that all non-negative integers n have a unique solution $a + 2b = n$.
\\ \\
For negative integers M that have no solutions in $S$ a similar proof holds, but instead generating the terms $S_{m+1} = 10N - M$ and $S_{m+2} = -5N + M$.
\\ \\
For any integer M that currently has no solution in S, we can always add two terms $S_{m+1}$ and $S_{m+2}$ such that $S_{m+1} + 2S_{m+2} = M$ that do not result in duplicated sums.
\\ \\
Thus, there does exist a subset $X$ of the integers such that for any integer $n$ there is exactly one solution to $a + 2b = n$ with $a, b \in X$.
\end{solution*}



\end{document}