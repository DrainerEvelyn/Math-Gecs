\documentclass[12pt]{article}
\usepackage[utf8]{inputenc}
\usepackage{latexsym,amsfonts,amssymb,amsthm,amsmath}
\usepackage{float}
\usepackage{caption}
\usepackage{marginnote}
\usepackage{tikz}
\usepackage{hyperref}

\setlength{\parindent}{0in}
\setlength{\oddsidemargin}{0in}
\setlength{\textwidth}{6.5in}
\setlength{\textheight}{8.8in}
\setlength{\topmargin}{0in}
\setlength{\headheight}{18pt}

\newtheorem*{answer*}{Answer}
\newtheorem*{solution*}{Solution}
\newtheorem{remark}{Remark}

\title{Weekly Homework 25}
\author{Math Gecs}
\date{August 22, 2024}

\begin{document}
\maketitle

\subsection*{Exercise 1}
Do there exist 16 three digit numbers, using only three different digits in all, so that the all numbers give different residues when divided by 16?\\

Source: 1998 JBMO Problem 4\\


\begin{solution*}
Let the three different digits be $a, b, c$.
\\ \\
If $a, b, c$ all have the same parity, then all sixteen numbers will also have the same parity. But then they can only cover at most 8 residues modulo 16, so they cannot have distinct residues by the pigeonhole principle.
\\ \\
Suppose that $a, b, c$ do not all have the same parity. If two are even and one is odd, then twelve of the eighteen possible three-digit numbers formed with $a, b, c$ will be even and six will be odd. But this means that there are not enough odd numbers to fill the 8 odd residues modulo 16, so it will not be possible to select 16 three digit numbers with this property in this case.
\\ \\
Similarly, if two are odd and one is even, then twelve of the eighteen possible three-digit numbers formed with $a, b, c$ will be odd and six will be even. But this means that there are not enough even numbers to fill the 8 even residues modulo 16, so it will not be possible to select 16 three digit numbers with this property in this case.
\\ \\
Therefore, we conclude that it is impossible to select 16 such numbers.
\end{solution*}


\end{document}