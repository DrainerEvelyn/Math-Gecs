\documentclass[12pt]{article}
\usepackage[utf8]{inputenc}
\usepackage{latexsym,amsfonts,amssymb,amsthm,amsmath}
\usepackage{float}
\usepackage{caption}
\usepackage{marginnote}
\usepackage{tikz}
\usepackage{hyperref}

\setlength{\parindent}{0in}
\setlength{\oddsidemargin}{0in}
\setlength{\textwidth}{6.5in}
\setlength{\textheight}{8.8in}
\setlength{\topmargin}{0in}
\setlength{\headheight}{18pt}

\newtheorem*{answer*}{Answer}
\newtheorem*{solution*}{Solution}
\newtheorem{remark}{Remark}

\title{Weekly Homework 43}
\author{Math Gecs}
\date{December 30, 2024}

\begin{document}
\maketitle

\subsection*{Exercise 1}
A moving particle starts at the point $(4,4)$ and moves until it hits one of the coordinate axes for the first time. When the particle is at the point $(a,b)$, it moves at random to one of the points $(a-1,b)$, $(a,b-1)$, or $(a-1,b-1)$, each with probability $\frac{1}{3}$, independently of its previous moves. The probability that it will hit the coordinate axes at $(0,0)$ is $\frac{m}{3^n}$, where $m$ and $n$ are positive integers such that $m$ is not divisible by $3$. Find $m + n$.\\

Source: 2019 AIME I Problem 5\\

\begin{solution*}
One could recursively compute the probabilities of reaching $(0,0)$ as the first axes point from any point $(x,y)$ as\[P(x,y) = \frac{1}{3} P(x-1,y) + \frac{1}{3} P(x,y-1) + \frac{1}{3} P(x-1,y-1)\]for $x,y \geq 1,$ and the base cases are $P(0,0) = 1, P(x,0) = P(y,0) = 0$ for any $x,y$ not equal to zero. We then recursively find $P(4,4) = \frac{245}{2187}$ so the answer is $245 + 7 = \boxed{252}$.
\end{solution*}


\end{document}