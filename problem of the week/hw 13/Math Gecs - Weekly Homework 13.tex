\documentclass[12pt]{article}
\usepackage[utf8]{inputenc}
\usepackage{latexsym,amsfonts,amssymb,amsthm,amsmath}
\usepackage{float}
\usepackage{caption}
\usepackage{marginnote}
\usepackage{tikz}
\usepackage{hyperref}

\setlength{\parindent}{0in}
\setlength{\oddsidemargin}{0in}
\setlength{\textwidth}{6.5in}
\setlength{\textheight}{8.8in}
\setlength{\topmargin}{0in}
\setlength{\headheight}{18pt}

\newtheorem*{answer*}{Answer}
\newtheorem*{solution*}{Solution}
\newtheorem{remark}{Remark}

\title{Weekly Homework 13}
\author{Math Gecs}
\date{April 22, 2024}

\begin{document}
\maketitle

\subsection*{Exercise 1}
If $a_1 \geq a_2 \geq \cdots \geq a_n \geq 0$ and $a_1+a_2+\cdots+a_n=1$, then prove:
$$a_1^2+3a_2^2+5a_3^2+ \cdots +(2n-1)a_n^2 \leq 1$$
\\

Source: 2002 Pan African MO Problem 6\\

\begin{solution*}
Note that $1 = (a_1 + a_2 + \cdots + a_n)^2 = \sum_{i=1}^{n} (a_i^2) + 2a_1a_2 + 2a_1a_3 + \cdots 2a_{n-1}a_n$.  Additionally, if $i \le j$, then $a_i \ge a_j$ and $2a_i a_j \ge 2a_j^2$.

$$$$

For a given value $j$, there are $j-1$ terms in the form $2a_ia_j$.  Thus,
$$\begin{align*}
(\sum_{i=1}^n a_i)^2 &\ge \sum_{i=1}^{n} (a_i^2) + 2a_2^2 + 4a_3^2 + \cdots + (2n-2)a_n^2 \\
1 &\ge a_1^2 + 3a_2^2 + 5a_3^2 + \cdots + (2n-1)a_n^2.
\end{align*}$$

\end{solution*}

\vspace{2in}






\subsection*{Exercise 2}
Let $a, b, c \geq 0$ and satisfy
$$a^2 + b^2 + c^2 + abc = 4.$$
Show that
$$0 \le ab + bc + ca - abc \leq 2.$$\\

Source: 2001 USAMO Problem 3\\

\begin{proof}[\proofname\ 1]
First we prove the lower bound.

Note that we cannot have $a, b, c$ all greater than 1.
Therefore, suppose $a \le 1$.
Then
$$ab + bc + ca - abc = a(b + c) + bc(1-a) \ge 0.$$
Note that, by the \href{https://artofproblemsolving.com/wiki/index.php/Pigeonhole_Principle}{Pigeonhole Principle}, at least two of $a,b,c$ are either both greater than or less than $1$. \href{https://artofproblemsolving.com/wiki/index.php/Without_loss_of_generality}{Without loss of generality}, let them be $b$ and $c$. Therefore, $(b-1)(c-1)\ge 0$. From the given equation, we can express $a$ in terms of $b$ and $c$ as
$$a=\frac{\sqrt{(4-b^2)(4-c^2)}-bc}{2}$$
Thus,
$$ab + bc + ca - abc = -a (b-1)(c-1)+a+bc \le a+bc = \frac{\sqrt{(4-b^2)(4-c^2)} + bc}{2}$$

From the \href{https://artofproblemsolving.com/wiki/index.php/Cauchy-Schwarz_Inequality}{Cauchy-Schwarz Inequality},
$$\frac{\sqrt{(4-b^2)(4-c^2)} + bc}{2} \le \frac{\sqrt{(4-b^2+b^2)(4-c^2+c^2)} }{2} = 2.$$

This completes the proof.
\end{proof}

$$$$

\begin{proof}[\proofname\ 2]
The proof for the lower bound is the same as in the first solution.

Now we prove the upper bound. Let us note that at least two of the three numbers $a$, $b$, and $c$ are both greater than or equal to 1 or less than or equal to 1. Without loss of generality, we assume that the numbers with this property are $b$ and $c$. Then we have
$$(1 - b)(1 - c)\geq 0.$$
The given equality $a^2 + b^2 + c^2 + abc = 4$ and the inequality $b^2 + c^2\geq 2bc$ imply that
$$a^2 + 2bc + abc\leq 4,$$
or
$$bc(2 + a)\leq 4 - a^2.$$
Dividing both sides of the last inequality by $2 + a$ yields
$$bc\leq 2 - a.$$
Thus,
$$ab + bc + ca - abc\leq ab + 2 - a + ac(1 - b) = 2 - a(1 + bc - b - c) = 2 - a(1 - b)(1 - c)\leq 2,$$
as desired.

The last equality holds if and only if $b = c$ and $a(1 - b)(1 - c) = 0$. Hence equality for the upper bound holds if and only if $(a,b,c)$ is one of the triples $(1,1,1)$, $(0,\sqrt{2},\sqrt{2})$, $(\sqrt{2},0,\sqrt{2})$, and $(\sqrt{2},\sqrt{2},0)$. Equality for the lower bound holds if and only if $(a,b,c)$ is one of the triples $(2,0,0)$, $(0,2,0)$ and $(0,0,2)$.
\end{proof}

$$$$

\begin{proof}[\proofname\ 3]
 
The proof for the lower bound is the same as in the first solution.

Now we prove the upper bound. It is clear that $a,b,c\leq 2$. If $abc = 0$, then the result is trivial. Suppose that $a,b,c > 0$. Solving for $a$ yields
$$a = \frac{-bc + \sqrt{b^2c^2 - 4(b^2 + c^2 - 4)}}{2} = \frac{-bc + \sqrt{(4 - b^2)(4 - c^2)}}{2}.$$
This asks for the trigonometric substitution $b = 2\sin u$ and $c = 2\sin v$, where $0^\circ < u,v < 90^\circ$. Then
$$a = 2(-\sin u\sin v + \cos u\cos v) = 2\cos (u + v),$$
and if we set $u = B/2$ and $v = C/2$, then $a = 2\sin (A/2)$, $b = 2\sin (B/2)$, and $c = \sin (C/2)$, where $A$, $B$, and $C$ are the angles of a triangle. We have
$$\begin{align*}
ab &= 4\sin\frac{A}{2}\sin\frac{B}{2} \\
&= 2\sqrt{\sin A\tan\frac{A}{2}\sin B\tan\frac{B}{2}} = 2\sqrt{\sin A\tan\frac{B}{2}\sin B\tan\frac{A}{2}} \\
&\leq \sin A\tan\frac{B}{2} + \sin B\tan\frac{A}{2} \\
&= \sin A\cot\frac{A + C}{2} + \sin B\cot\frac{B + C}{2},
\end{align*}$$
where the inequality step follows from AM-GM. Likewise,
$$\begin{align*}
bc &\leq \sin B\cot\frac{B + A}{2} + \sin C\cot\frac{C + A}{2}, \\
ca &\leq \sin A\cot\frac{A + B}{2} + \sin C\cot\frac{C + B}{2}.
\end{align*}$$
Therefore
$$\begin{align*}
ab + bc + ca &\leq (\sin A + \sin B)\cot\frac{A + B}{2} + (\sin B + \sin C)\cot\frac{B + C}{2} + (\sin C + \sin A)\cot\frac{C + A}{2} \\
&= 2\left(\cos\frac{A - B}{2}\cos\frac{A + B}{2} + \cos\frac{B - C}{2}\cos\frac{B + C}{2} + \cos\frac{C - A}{2}\cos\frac{C + A}{2} \right)\\
&= 2(\cos A + \cos B + \cos C) \\
&= 6 - 4\left(\sin^2\frac{A}{2} + \sin^2\frac{B}{2} + \sin^2\frac{C}{2}\right) \\
&= 6 - (a^2 + b^2 + c^2) = 2 + abc,
\end{align*}$$
as desired.   
\end{proof}

\end{document}