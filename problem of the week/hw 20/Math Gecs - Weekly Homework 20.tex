\documentclass[12pt]{article}
\usepackage[utf8]{inputenc}
\usepackage{latexsym,amsfonts,amssymb,amsthm,amsmath}
\usepackage{float}
\usepackage{caption}
\usepackage{marginnote}
\usepackage{tikz}
\usepackage{hyperref}

\setlength{\parindent}{0in}
\setlength{\oddsidemargin}{0in}
\setlength{\textwidth}{6.5in}
\setlength{\textheight}{8.8in}
\setlength{\topmargin}{0in}
\setlength{\headheight}{18pt}

\newtheorem*{answer*}{Answer}
\newtheorem*{solution*}{Solution}
\newtheorem{remark}{Remark}

\title{Weekly Homework 20}
\author{Math Gecs}
\date{June 16, 2024}

\begin{document}
\maketitle

\subsection*{Exercise 1}
There are exactly $K$ positive integers $b$ with $5 \leq b \leq 2024$ such that the base-$b$ integer $2024_b$ is divisible by $16$ (where $16$ is in base ten). What is the sum of the digits of $K$?\\

Source: 2024 AMC 10A Problem 18\\

\begin{answer*}
$\boxed{20}$
\end{answer*}

\begin{solution*}
$2b^3+2b+4\equiv 0\pmod{16}\implies b^3+b+2\equiv 0\pmod 8$, if $b$ even then $b+2\equiv 0\pmod 8\implies b\equiv 6\pmod 8$. If $b$ odd then $b^2\equiv 1\pmod 8\implies b^3+b+2\equiv 2b+2\pmod 8$ so $2b+2\equiv 0\pmod 8\implies b+1\equiv 0\pmod 4\implies b\equiv 3,7\pmod 8$. Now $8\mid 2024$ so $\frac38\cdot 2024=759$ but $3$ is too small so $759 - 1 = 758\implies\boxed{\textbf{(D) }20}$.
\end{solution*}

\end{document}