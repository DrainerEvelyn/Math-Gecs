\documentclass[12pt]{article}
\usepackage[utf8]{inputenc}
\usepackage{latexsym,amsfonts,amssymb,amsthm,amsmath}
\usepackage{float}
\usepackage{caption}
\usepackage{marginnote}
\usepackage{tikz}
\usepackage{hyperref}

\setlength{\parindent}{0in}
\setlength{\oddsidemargin}{0in}
\setlength{\textwidth}{6.5in}
\setlength{\textheight}{8.8in}
\setlength{\topmargin}{0in}
\setlength{\headheight}{18pt}

\newtheorem*{answer*}{Answer}
\newtheorem*{solution*}{Solution}
\newtheorem{remark}{Remark}

\title{Weekly Homework 22}
\author{Math Gecs}
\date{June 30, 2024}

\begin{document}
\maketitle

\subsection*{Exercise 1}
Determine the remainder obtained when the expression\[2004^{2003^{2002^{2001}}}\]is divided by $1000$.\\

Source: Mock AIME 2 Pre 2005 Problems/Problem 8\\

\begin{solution*}
We note that $2004^{2003^{2002^{2001}}} \equiv 4^{2003^{2002^{2001}}}\pmod{1000}$. The remainder of the RHS modulo $8$ is trivially zero, but the remainder of the RHS modulo $125$ depends on the remainder of the exponent modulo $\phi(125) = 50$, so we defer the calculation until later.
\\ \\
We compute $2003^{2002^{2001}}$ modulo $50$; again noting that this is equivalent to $3^{2002^{2001}}$ modulo $50$. The remainder is trivially one modulo two, but the remainder modulo $25$ depends on the remainder of the second exponent modulo $\phi(25) = 20$.
\\ \\
Now we start to unroll the recursion: We have $2002^{2001} \equiv 2^{2001} \pmod{20}$. Modulo four, the remainder is trivially zero; modulo five, the remainder is $2^{2001\pmod{5}} \equiv 2^1 \equiv 2\pmod{5}$, so we have $2002^{2001} \equiv 12\pmod{20}$.
\\ \\
Then $2003^{2002^{2001}} \equiv 3^{12} \equiv 16 \pmod{25}$, so that $2003^{2002^{2001}} \equiv 41\pmod{50}$ by CRT.
\\ \\
Then $2004^{2003^{2002^{2001}}} \equiv 4^{41} \equiv 78\pmod{125}$, so that $2004^{2003^{2002^{2001}}} \equiv 704 \pmod{1000}$ by CRT, and we are done.
\end{solution*}


\end{document}