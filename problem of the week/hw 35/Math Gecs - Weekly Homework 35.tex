\documentclass[12pt]{article}
\usepackage[utf8]{inputenc}
\usepackage{latexsym,amsfonts,amssymb,amsthm,amsmath}
\usepackage{float}
\usepackage{caption}
\usepackage{marginnote}
\usepackage{tikz}
\usepackage{hyperref}

\setlength{\parindent}{0in}
\setlength{\oddsidemargin}{0in}
\setlength{\textwidth}{6.5in}
\setlength{\textheight}{8.8in}
\setlength{\topmargin}{0in}
\setlength{\headheight}{18pt}

\newtheorem*{answer*}{Answer}
\newtheorem*{solution*}{Solution}
\newtheorem{remark}{Remark}

\title{Weekly Homework 35}
\author{Math Gecs}
\date{November 05, 2024}

\begin{document}
\maketitle

\subsection*{Exercise 1}
In the polynomial $x^4 - 18x^3 + kx^2 + 200x - 1984 = 0$, the product of $2$ of its roots is $- 32$. Find $k$.\\

Source: 1984 USAMO Problem 1\\

\begin{answer*}
$\boxed{86}$
\end{answer*}

\begin{solution*}
Using Vieta's formulas, we have:
\\ \\
\begin{align*}a+b+c+d &= 18,\\ ab+ac+ad+bc+bd+cd &= k,\\ abc+abd+acd+bcd &=-200,\\ abcd &=-1984.\\ \end{align*}
\\ \\
From the last of these equations, we see that $cd = \frac{abcd}{ab} = \frac{-1984}{-32} = 62$. Thus, the second equation becomes $-32+ac+ad+bc+bd+62=k$, and so $ac+ad+bc+bd=k-30$. The key insight is now to factor the left-hand side as a product of two binomials: $(a+b)(c+d)=k-30$, so that we now only need to determine $a+b$ and $c+d$ rather than all four of $a,b,c,d$.
\\ \\
Let $p=a+b$ and $q=c+d$. Plugging our known values for $ab$ and $cd$ into the third Vieta equation, $-200 = abc+abd + acd + bcd = ab(c+d) + cd(a+b)$, we have $-200 = -32(c+d) + 62(a+b) = 62p-32q$. Moreover, the first Vieta equation, $a+b+c+d=18$, gives $p+q=18$. Thus we have two linear equations in $p$ and $q$, which we solve to obtain $p=4$ and $q=14$.
\\ \\
Therefore, we have $(\underbrace{a+b}_4)(\underbrace{c+d}_{14}) = k-30$, yielding $k=4\cdot 14+30 = \boxed{86}$.
\end{solution*}


\end{document}