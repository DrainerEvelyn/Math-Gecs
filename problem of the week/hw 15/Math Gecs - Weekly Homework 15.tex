\documentclass[12pt]{article}
\usepackage[utf8]{inputenc}
\usepackage{latexsym,amsfonts,amssymb,amsthm,amsmath}
\usepackage{float}
\usepackage{caption}
\usepackage{marginnote}
\usepackage{tikz}
\usepackage{hyperref}

\setlength{\parindent}{0in}
\setlength{\oddsidemargin}{0in}
\setlength{\textwidth}{6.5in}
\setlength{\textheight}{8.8in}
\setlength{\topmargin}{0in}
\setlength{\headheight}{18pt}

\newtheorem*{answer*}{Answer}
\newtheorem*{solution*}{Solution}
\newtheorem{remark}{Remark}

\title{Weekly Homework 15}
\author{Math Gecs}
\date{May 07, 2024}

\begin{document}
\maketitle

\subsection*{Exercise 1}
For each positive integer $n$, let $S(n)$ be the number of sequences of length $n$ consisting solely of the letters $A$ and $B$, with no more than three $A$s in a row and no more than three $B$s in a row. What is the remainder when $S(2015)$ is divided by $12$?
\\ \\
Source: 2015 AMC 12A Problem 22

\begin{answer*}
$\boxed {8}$
\end{answer*}

\begin{solution*}
We can start off by finding patterns in $S(n)$. When we calculate a few values we realize either from performing the calculation or because the calculation was performed in the exact same way that $S(n) = 2^n - 2((n_4)- (n_5) \dots (n_n))$. Rearranging the expression we realize that the terms aside from $2^{2015}$ are congruent to $0$ mod $12$(Just put the equation in terms of $2^{2015}$ and the four combinations excluded and calculate the combinations mod $12$). Using patterns we can see that $2^{2015}$ is congruent to $8$ mod $12$. Therefore $\boxed {8}$ is our answer. ~Very minor edit in LaTeX by get-rickrolled
\end{solution*}






\end{document}