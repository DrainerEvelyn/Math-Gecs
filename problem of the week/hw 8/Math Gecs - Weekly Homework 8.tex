\documentclass[12pt]{article}
\usepackage[utf8]{inputenc}
\usepackage{latexsym,amsfonts,amssymb,amsthm,amsmath}
\usepackage{float}
\usepackage{caption}
\usepackage{marginnote}
\usepackage{tikz}
\usepackage{hyperref}

\setlength{\parindent}{0in}
\setlength{\oddsidemargin}{0in}
\setlength{\textwidth}{6.5in}
\setlength{\textheight}{8.8in}
\setlength{\topmargin}{0in}
\setlength{\headheight}{18pt}

\newtheorem*{answer*}{Answer}
\newtheorem*{solution}{Solution}
\newtheorem{remark}{Remark}

\title{Weekly Homework 8}
\author{Math Gecs}
\date{March 16, 2024}

\begin{document}
\maketitle

\subsection*{Exercise 1}
Let $x$ and $y$ be positive reals such that $$ x^3 + y^3 + (x + y)^3 + 30xy = 2000. $$ Show that $x + y = 10$.\\

Source: 2000 JBMO Problem 1

\begin{proof}
Rearranging the equation yields
$$x^3 + y^3 + (x + y)^3 + 30xy - 2000 = 0.$$
If $x+y=10$ in the large equation, then $x+y-10$ must be a [[factor]] of the large equation.  Note that we can rewrite the large equation as
$$\begin{align*}
0 &= (x+y)^3 - 1000 + x^3 + 3x^2y - 3x^2y + 3xy^2 - 3xy^2 + y^3 - 1000 + 30xy \\
&= 2[(x+y)^3 - 1000] - 3x^2y - 3xy^2 + 30xy.
\end{align*}$$
We can factor the difference of cubes in the first part and factor $3xy$ in the second part, resulting in
$$0 = 2(x+y-10)((x+y)^2 + 10x + 10y + 100) - 3xy(x+y-10)$$
Finally, we can factor by grouping, which results in
$$\begin{align*}
0 &= (x+y-10)(2(x+y)^2 + 20x + 20y + 200 - 3xy) \\
&= (x+y-10)(2x^2 + xy + 2y^2 + 20x + 20y + 200).
\end{align*}$$
By the Zero Product Property, either $x+y=10$ or $2x^2 + xy + 2y^2 + 20x + 20y + 200 = 0.$  However, since $x$ and $y$ are both positive, $2x^2 + xy + 2y^2 + 20x + 20y + 200$ can not equal zero, so we have proved that $x+y = 10.$

\end{proof}

\vspace{2in}






\subsection*{Exercise 2}
Let $a, b, c \geq 0$ and satisfy
$$a^2 + b^2 + c^2 + abc = 4.$$
Show that
$$0 \le ab + bc + ca - abc \leq 2.$$


Source: 2001 USAMO Problem 3

$$$$

\begin{proof}
First we prove the lower bound.

Note that we cannot have $a, b, c$ all greater than 1.
Therefore, suppose $a \le 1$.
Then
$$ab + bc + ca - abc = a(b + c) + bc(1-a) \ge 0.$$
Note that, by the \href{https://artofproblemsolving.com/wiki/index.php/Pigeonhole_Principle}{Pigeonhole Principle}, at least two of $a,b,c$ are either both greater than or less than $1$. \href{https://artofproblemsolving.com/wiki/index.php/Without_loss_of_generality}{Without loss of generality}, let them be $b$ and $c$. Therefore, $(b-1)(c-1)\ge 0$. From the given equation, we can express $a$ in terms of $b$ and $c$ as
$$a=\frac{\sqrt{(4-b^2)(4-c^2)}-bc}{2}$$
Thus,
$$ab + bc + ca - abc = -a (b-1)(c-1)+a+bc \le a+bc = \frac{\sqrt{(4-b^2)(4-c^2)} + bc}{2}$$

From the \href{https://artofproblemsolving.com/wiki/index.php/Cauchy-Schwarz_Inequality}{Cauchy-Schwarz Inequality},
$$\frac{\sqrt{(4-b^2)(4-c^2)} + bc}{2} \le \frac{\sqrt{(4-b^2+b^2)(4-c^2+c^2)} }{2} = 2.$$

This completes the proof.
\end{proof}


\end{document}