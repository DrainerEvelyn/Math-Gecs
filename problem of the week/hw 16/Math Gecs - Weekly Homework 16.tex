\documentclass[12pt]{article}
\usepackage[utf8]{inputenc}
\usepackage{latexsym,amsfonts,amssymb,amsthm,amsmath}
\usepackage{float}
\usepackage{caption}
\usepackage{marginnote}
\usepackage{tikz}
\usepackage{hyperref}

\setlength{\parindent}{0in}
\setlength{\oddsidemargin}{0in}
\setlength{\textwidth}{6.5in}
\setlength{\textheight}{8.8in}
\setlength{\topmargin}{0in}
\setlength{\headheight}{18pt}

\newtheorem*{answer*}{Answer}
\newtheorem*{solution*}{Solution}
\newtheorem{remark}{Remark}

\title{Weekly Homework 16}
\author{Math Gecs}
\date{May 14, 2024}

\begin{document}
\maketitle

\subsection*{Exercise 1}
How many sequences of $0$s and $1$s of length $19$ are there that begin with a $0$, end with a $0$, contain no two consecutive $0$s, and contain no three consecutive $1$s?
\\\\
Source: 2019 AMC 10B Problem 25

\begin{answer*}
$\boxed{65}$
\end{answer*}

\begin{solution*}
Let $f(n)$ be the number of valid sequences of length $n$ (satisfying the conditions given in the problem).

We know our valid sequence must end in a $0$. Then, since we cannot have two consecutive $0$s, it must end in a $10$. Now, we only have two cases: it ends with $010$, or it ends with $110$ which is equivalent to $0110$. Thus, our sequence must be of the forms $0\ldots010$ or $0\ldots0110$. In the first case, the first $n-2$ digits are equivalent to a valid sequence of length $n-2$. In the second, the first $n-3$ digits are equivalent to a valid sequence of length $n-3$. Therefore, it must be the case that $f(n) = f(n-3) + f(n-2)$, with $n \ge 3$ (because otherwise, the sequence would contain only 0s and this is not allowed due to the given conditions).
\\ \\
It is easy to find $f(3) = 1$ since the only possible valid sequence is $010$. $f(4)=1$ since the only possible valid sequence is $0110$. $f(5)=1$ since the only possible valid sequence is $01010$.
\\ \\
The recursive sequence is then as follows:

\[f(3)=1\]\[f(4)=1\]\[f(5) = 1\]\[f(6) = 1 + 1 = 2\]\[f(7) = 1 + 1 = 2\]\[f(8) = 1 + 2 = 3\]\[f(9) = 2 + 2 = 4\]\[f(10) = 2 + 3 = 5\]\[f(11) = 3 + 4 = 7\]\[f(12) = 4 + 5 = 9\]\[f(13) = 5 + 7 = 12\]\[f(14) = 7 + 9 = 16\]\[f(15) = 9 + 12 = 21\]\[f(16) = 12 + 16 = 28\]\[f(17) = 16 + 21 = 37\]\[f(18) = 21 + 28 = 49\]\[f(19) = 28 + 37 = 65\]

So, our answer is $\boxed{65}$.
\end{solution*}


\end{document}